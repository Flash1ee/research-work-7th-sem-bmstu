\chapter*{ЗАКЛЮЧЕНИЕ}
\addcontentsline{toc}{chapter}{ЗАКЛЮЧЕНИЕ} 

В ходе выполнения данной работы были выполнены следующие задачи:
\begin{itemize}
\item Проведен обзор предметной области и описаны ее термины.
\item Классифицированы форматы сериализации данных.
\item Выявлены критерии сравнения.
\item Проведено сравнение форматов.
\end{itemize}
 
Все поставленные задачи были решены. Цель данной работы была достигнута.

Исследование показало, что для наборов данных, состоящих из чисел и строк, лучше всего с сериализацией справляется Protobuf. Но с ним сложно работать из-за наличия схемы, в которой нужно описывать все передаваемые структуры и затем генерировать код, который нужно интегрировать в приложение.
В качестве альтернативы следует использовать CBOR, который чуть менее производительный, но более легкий в разработке формат сериализации.  

