\chapter{Основная часть}

%\section{Характеристика организации}
%Организация, в которой проходила практика - ООО "Авито Тех".
%
%Основная деятельность - разработка интернет-сервиса для размещения объявлений о товарах, недвижимости, вакансиях и резюме на рынке труда, а также услугах от частных лиц и компаний, продажи бытовой техники, электроники, одежды.
%Авито занимает первое место в мире среди классифайдов по посещаемости в 2021 году \cite{avito-top-classified}.  
%
%Для прохождения практики был определен в "Департамент разработки Goods Classified"\ , юнит "Goods Horizontal Solutions"\ , команду разработки "LeGo (IMV)".
% 
%Команда занимается разработкой сервиса оценки рыночной стоимости для объявлений в категории "Товары"\ и добавление расширенных состояний.  
%
%\section{Характеристика проделанной работы}
%\subsection{Проектирования таблиц базы данных для хранения обратной связи пользователей}
%
%Для сохранения обратной связи понадобятся следующие таблицы:
%\begin{enumerate}
%\item Feedback - оставленная пользователями оценка;
%\item Slugs - хранение идентификаторов выбранного ответа пользователя;
%\end{enumerate}
%
%\noindent\textbf{Таблица Feedback}\\
%Таблица Feedback содержит информацию об ответах пользователей (описана часть полей):
%\begin{table}[!ht]
%    \caption{Описание полей таблицы \texttt{Feedback}}
%    \label{tbl:feedback}
%    \begin{center}
%        \begin{tabular}{|p{0.3\textwidth}p{0.6\textwidth}|}
%            \hline
%            \textbf{Поле} & \textbf{Значение} \\\hline
%            \texttt{id} & Идентификатор записи, уникальный. \\\hline
%%            \texttt{product\_id} & Идентификатор объявления\\\hline
%            \texttt{source\_price} & Цена, указанная в объявлении \\\hline
%            
%            \texttt{user\_id} & Идентификатор пользователя \\\hline
%            \texttt{estimated} & Рассчитанная рыночная стоимость (IMV) \\\hline
%            \texttt{agree} & Уникальный идентификатор, обозначающий ответ пользователя (Да/Нет)\\\hline
%%            \texttt{user\_comment} & Комментарий \\\hline
%            \texttt{created\_at} & Время отправки первого отзыва \\\hline
%            \texttt{updated\_at} & Время последнего обновления отзыва \\\hline
%        \end{tabular}
%    \end{center}
%\end{table}
%
%\newpage
%\noindent\textbf{Таблица Slugs}\\
%Таблица Slugs содержит информацию об идентификаторах ответов пользователей:
%\begin{table}[!ht]
%    \caption{Описание полей таблицы \texttt{Slugs}}
%    \label{tbl:slugs}
%    \begin{center}
%        \begin{tabular}{|p{0.3\textwidth}p{0.6\textwidth}|}
%            \hline
%            \textbf{Поле} & \textbf{Значение} \\\hline
%            \texttt{id} & Идентификатор записи, уникальный. \\\hline
%            \texttt{feedback\_id} & Внешний ключ для связи с таблицей Feedback\\\hline
%            \texttt{slug} & Уникальный идентификатор, признак варианта ответа пользователя (при выборе Нет) \\\hline
%            \texttt{created\_at} & Время отправки отзыва \\\hline
%        \end{tabular}
%    \end{center}
%\end{table}
%
%ER-модель сущностей представлена ниже:
%\imgw{sql-table}{h!}{0.7\textwidth}{ER диаграмма сущностей}
%
%\newpage
%\subsection{Бизнес логика и конечная точка для формы обратной связи}
%
%Форма обратной связи используется на всех версиях сайта:
%\begin{itemize}
%\item для ПК;
%\item версия для мобильных устройств;
%\item приложения для Android \cite{android} и IOS \cite{ios}.
%\end{itemize}
%
%Из-за различий в технических возможностях устройств (в Авито мобильные устройства не могут отправлять тело запроса в формате application/json из-за используемых библиотек) , было принято решение реализовать две конечные точки.  
%
%
%
%Для версии ПК (мобильная и обычная) - /api/send, тип передаваемого контента application/json.    
%
%Для приложений на Android и IOS - /web/send, тип передаваемого контента x-www-form-urlencoded;  
%
%Бизнес логика для обоих обработчиков идентична, достаточно реализовать один раз и переиспользовать.  
%
%
%Для документирования API использовался Swagger \cite{swagger}. Данный инструмент визуализирует документацию, составленную в соответствии со стандартом OpenApi 3 \cite{openapi}. 
%\newpage
%Пример документации представлен ниже:
%\imgw{swagger_all}{h!}{1.0\textwidth}{Документация Swagger OpenAPI 3}
%\newpage
%\subsection{Добавление метрик и настройка дашборда}
%
%Метрика программного обеспечения (software metric) – численная мера, позволяющая оценить определенные свойства конкретного участка программного кода \cite{metrics}.
%
%Для реализованной конечной точки были добавлены метрики, позволяющие построить графики и  отслеживать действия пользователей.  
%
%Для построения дашбордов с графиками использовалась система визуализации данных Grafana \cite{grafana}.  
%
%На дашборде есть несколько графиков:
%\begin{itemize}
%\item \textit{estimation\_not\_found}, \textit{item\_err\_response\_data} и \textit{item\_err}  - ошибки при проверке и агрегации данных;
%\item \textit{exists} - пользователь уже оставлял отзыв ранее, произошло обновление ранее оставленного;
%\item \textit{new} - пользователь оставлил новый отзыв. 
%\end{itemize}
%
%Графики представлены ниже:
%\imgr{grafana_dashboard}{h!}{90}{0.37}{Визуализация взаимодействия пользователей и конечной точки}
%
%\newpage
%\section{Демонстрация работы}
%
%\subsection{Пользовательский опыт}
%
%На рисунке ниже представлен функционал до добавления формы обратной свзяи.  
%\imgw{scale_before}{h!}{0.8\textwidth}{Шкала оценки рыночной цены до изменений}
%
%На рисунке ниже представлен функционал после добавления формы обратной свзяи.  
%\imgw{scale_after}{h!}{0.8\textwidth}{Шкала оценки рыночной цены после изменений}
%\newpage
%Пользователь нажал на кнопку "Да":
%\imgw{send_yes_feedback}{h!}{0.8\textwidth}{Пользователь согласен с оценкой цены}
%
%Пользователь нажал на кнопку "Нет":
%\imgw{send_no_feedback_2}{h!}{0.8\textwidth}{Пользователь не согласен с оценкой цены}
%
%\newpage
%\subsection{Серверная часть}
%Пользователь нажал на кнопку "Да".
%
%Клиент отправляет следующий запрос:
%\begin{lstlisting}
%curl -X POST --location "https://avito.ru/web/send" \
%    -H "Content-Type: application/json" \
%    -d "{
%          \"id\": 75357893,
%          \"slug\": \"yes\"
%        }"
%\end{lstlisting}
%
%Ответ сервера:
%\begin{lstlisting}
%HTTP/1.1 200 OK
%Server: nginx
%Content-Type: application/json
%Content-Length: 28
%Connection: keep-alive
%Keep-Alive: timeout=75
%x-envoy-upstream-service-time: 23
%X-XSS-Protection: 1; mode=block
%X-Content-Type-Options: nosniff
%
%{
%  "status": "ok"
%}
%\end{lstlisting}
%\newpage
%Пользователь нажал на кнопку "Нет".
%
%Клиент отправляет следующий запрос:
%\begin{lstlisting}
%curl -X POST --location "https://avito.ru/web/send" \
%    -H "Content-Type: application/json" \
%     -d "{
%          \"id\": 2649475263,
%          \"slug\": \"no\",
%          \"values\": [
%            \"bad_range\",
%            \"bad_description\"
%          ]
%        }"
%\end{lstlisting}
%
%Ответ сервера:
%\begin{lstlisting}
%HTTP/1.1 200 OK
%Server: nginx
%Content-Type: application/json
%Content-Length: 140
%Connection: keep-alive
%Keep-Alive: timeout=75
%x-envoy-upstream-service-time: 14
%X-XSS-Protection: 1; mode=block
%X-Content-Type-Options: nosniff
%
%{
%  "status": "ok",
%  "data": {
%    "title": "Thank you for the answer!",
%    "description": "We'll see what can be done"
%  }
%}
%\end{lstlisting}