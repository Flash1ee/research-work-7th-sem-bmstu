\chapter*{ВВЕДЕНИЕ}
\addcontentsline{toc}{chapter}{ВВЕДЕНИЕ}

По данным за 2020 год в России интернет пользователями являются 118 миллионов человек. Всего на 2020 год в России проживало 145.9 миллионов человек, это значит, что пользуется интернетом 81 \% от всего населения.  

Основными устройствами являются компьютеры и мобильные устройства -- смартфоны. На них приходится 86 \% и 91 \% соответственно.

Большую часть времени пользователи используют браузер.
В современном вебе подавляющее большинство сайтов являются веб приложениями. Поэтому пользователи являются клиентами, которые взаимодействует с сервером при помощи браузера.

На серверной стороне приложения используют протоколы, позволяющие взаимодействовать между собой. 

Один из используемых протоколов взаимодействия - RPC.

RPC - это удаленный вызов процедур. Реализация протокола включает в себя два компонента: сетевой протокол для обмена данными по сети - транспорт и язык сериализации.  

Реализации RPC в качестве сетевого протокола используют TCP, UDP или HTTP.  
В качестве формата сериализации используют JSON или XML.

Цель работы --- проанализировать существующие методы сериализации данных.

Для достижения поставленной цели потребуется:
\begin{itemize}
\item Описать термины предметной области;
\item Описать анализируемые форматы сериализации данных;
\item Выявить критерии сравнения и сравнить форматы;
\end{itemize}
